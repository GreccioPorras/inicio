\documentclass{article}
\usepackage{hyperref}

\title{\textbf{Acerca del sitio web}}
\date{6 agosto del 2023}

\begin{document}

\maketitle

\section{Diseño y filosofía del sitio web}
El sitio web utiliza el paquete \emph{tutors}, que permite su diseño
y mantenimiento con mínimo conocimiento técnico, de manera tal que
cualquier estudiante pueda colaborar con la adición de contenido. En
lugar de escribir el sitio directamente en \emph{HTML}, solo se debe
escribir prácticamente en texto llano \emph{markdown}, el cual es
compilado a HTML. El proyecto cuenta con documentación comprensiva
para que futuros estudiantes de un área afín puedan colaborar con la
adición de material al sitio web.\\

Se hace uso del sistema de versionado de archivos \emph{git} para
evitar la pérdida accidental de trabajo y cambios, y los archivos son
sincronizados en \emph{GitHub}, una plataforma de colaboración de
código. GitHub no solo permite una colaboración abierta del sitio
web, sino que también permite la ejecución automática de acciones de
compilación y despliegue del en sus servidores.\\

Se optó porque el sitio web fuera estático, debido a que permite una
mayor flexibilidad de alojamiento y consulta local.

\section{Planeando hacia el futuro}

Los proyectos informáticos elaborados a nivel universitario tienen
antecedentes de ser abandonados una vez que sus creadores y
responsables de mantenerlos se gradúan o completan el proyecto
asociado. Por ello, creemos en el uso de tecnologías de código
abierto, como históricamente se ha hecho en la Universidad de Costa
Rica, y en la creación de documentación completa y exhaustiva.\\

\end{document}
